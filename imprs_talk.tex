\documentclass{beamer}
\usepackage{listings}
\usepackage{algpseudocode}
\usepackage{stmaryrd}
\usetheme{Frankfurt}
\beamertemplatenavigationsymbolsempty
\lstset{showstringspaces=false, frame=single}
\title{Soundness of Information Flow Typing in Coq}
\author{Adam Smith}
\date{February 9, 2015}
\begin{document}
  \frame{\titlepage}
  \section{Type Systems}
  \subsection{Type Systems}
  \begin{frame}
    \frametitle{Type Systems}
    \begin{itemize}
      \item ``A type system is a syntactic method for enforcing levels of
             abstraction in programs.'' - Pierce (2002)
      \pause
      \item Statically (at compile-time) ensure that operations are not
            performed on inappropriate data
      \pause
      \item By restricting types of values, we can prevent them from
            representing invalid states
      \pause
      \item Performance benefits
      \pause
      \item C/C++/Java programmers should be familiar with static type systems
      \pause
      \item Languages like Python, Perl, Ruby, PHP, Scheme/Lisp have no static
            type system to speak of
    \end{itemize}
  \end{frame}
\begin{frame}[fragile]
    \frametitle{Type Systems In Action}
    ``Statically (at compile-time) ensure that operations are not performed
      on inappropriate data''
    \begin{lstlisting}[language=C,
                       captionpos=b,
                       caption=A C program with a type error]
      int add(int a, int b) {
        return a + b;
      }
      add("Hello ", "world?");
    \end{lstlisting}
    \pause
    G++ says ``\texttt{error: invalid conversion from `const char*' to `int'}''
\end{frame}
\begin{frame}[fragile]
  \frametitle{Subtyping}
  Informally: Subtyping allows multiple types to be used in the same context,
  as long as they all adhere to some interface:
  \begin{lstlisting}[language=Java,
                     captionpos=b,
                     caption=Subtyping in the Java programming language]
    class A {
      void foo() { /*...*/ }
    }

    class B extends A {
      void foo() { /*...*/ }
    }

    void test(A a) { a.foo(); }

    B b = new B();
    test(b);
  \end{lstlisting}
\end{frame}
\section{Information Flow Security}
\subsection{Information Flow Security}
\begin{frame}
  \frametitle{What Is Information Flow Security?}
  \begin{itemize}
    \item Ensures that confidential data isn't accidentally or maliciously
          leaked
    \pause
    \item Distinct from access control
    \pause
    \item Can be checked with types
  \end{itemize}
  \pause
  \begin{definition}
    \emph{Non-interference}: in any two runs of the program with the same
    public inputs, the public output must be the same
  \end{definition}
\end{frame}
\begin{frame}[fragile]
  \frametitle{Example (From Sabelfeld \& Myers)}
  \begin{algorithmic}
    \State $h \gets h \mod 2$
    \State $l \gets 0$
    \If {$h = 1$}
      \State $l \gets 1$
    \EndIf
  \end{algorithmic}
  \pause
  \vspace{5mm}
  The final value of the public $l$ variable depends on the parity of the
  private $h$ variable; hence non-interference is not satisfied.
\end{frame}
\begin{frame}
  \frametitle{Checking Information Flow}
  \begin{itemize}
    \item Most common approach: define a new security-typed language and
          prove proerties about it
    \pause
    \item See: Volpano, Smith, Irvine (1996), Heintze \& Riecke (1998),
          Pottier \& Simonet (2002)...
    \pause
    \item Lots of work reproving basic correctness properties each time
    \pause
    \item A totally new system/language isn't as useful as extending
          existing ones
    \pause
    \item Tends to be very heavyweight
  \end{itemize}
\end{frame}
\section{Information Flow Inference For Free}
\subsection{Information Flow Inference For Free}
\begin{frame}
  \frametitle{Information Flow Inference For Free}
  \begin{itemize}
    \item Following Pottier \& Conchon (2000)
    \pause
    \item Compile a security-typed language into a traditional ML-like language
    \pause
    \item Security errors manifest as type errors
    \pause
    \item We get inference and correctness of the whole system ``for free''
  \end{itemize}
\end{frame}
\begin{frame}
  \frametitle{The Source Calculus}
  We start with a standard $\lambda$-calculus, extended with labels.
  \pause
  \begin{itemize}
    \item $\beta$-reduction: $(\lambda x. e1)\; e2 \rightarrow e1[e2/x]$
    \item Lifting labelled terms: $(l : e1)\; e2 \rightarrow l : (e1\; e2)$
  \end{itemize}
  The set of labels $L$ is arbitrary.
  \pause
  \begin{example}
    $(L : (\lambda x y. y)) (H : 27) \rightarrow^* L : \lambda y. y$
  \end{example}
\end{frame}
\begin{frame}
  \frametitle{The Source Calculus - Stability}
  \begin{theorem}
  Stability: ``given a computation sequence $e \rightarrow^* f$...$f$ does not
               depend on any subterm of $e$ which carries a label not found
               in $f$''
  \end{theorem}
  \pause
  \begin{example}
    \begin{itemize}
      \item Recall $(L : (\lambda x y. y)) (H : 27) \rightarrow^*
                     L : \lambda y. y$
      \pause
      \item Doesn't depend on the H value!
    \end{itemize}
  \end{example}
  \pause
  In general, non-interference: an L-labelled expression's value
  cannot depend on H-labelled data
\end{frame}
\begin{frame}
  \frametitle{Translating The Source Calculus}
  \begin{itemize}
    \item Translate source calculus to a (theoretically existing) ML-like
          language
    \pause
    \item Labels as terms \emph{and} types
    \pause
    \item Each source calculus expression paired with its highest label
  \end{itemize}
  \pause
  \begin{example}
    $\llparenthesis H : (L : 27) \rrparenthesis = \langle 27, H \rangle$ with
    type $int * H$.
  \end{example}
  \pause
  \begin{itemize}
    \item Organize labels in an upper semi-lattice; derive subtyping relation
          from lattice structure
    \pause
    \item Use type annotations to define information flow policy
  \end{itemize}
\end{frame}
\begin{frame}
  \frametitle{The Translation Scheme - Simulation}
  \begin{theorem}
    Simulation: if $e \rightarrow f$,
                then $\llparenthesis e \rrparenthesis \rightarrow^*
                       \llparenthesis f \rrparenthesis$
  \end{theorem}
  \begin{example}
    Recall $L : (\lambda x y. y) (H : 27) \rightarrow L : \lambda y. y$
           (abbreviate as $e \rightarrow f$)
   \begin{align*}
     \llparenthesis e \rrparenthesis =
       \langle & \text{fst } ((\lambda x. \langle \lambda y. \text{fst } y,
                                                  \epsilon\rangle)
                              \langle 27, H \rangle)  \\
               & L\; @\; \text{snd } ((\lambda x. \langle \lambda y.
                                                     \text{fst } y,
                                                          \epsilon\rangle)
                          \langle 27, H \rangle) \rangle\\
           \llparenthesis f \rrparenthesis =
                  \langle & \lambda y. \langle \text{fst }y,
                            \text{snd }y \rangle , L \rangle
   \end{align*}
   By simulation, $\llparenthesis e \rrparenthesis \rightarrow^*
                   \llparenthesis f \rrparenthesis$
  \end{example}
\end{frame}
\section{Formalization}
\subsection{Formalization}
\begin{frame}
  \frametitle{The Coq Theorem Prover}
  \begin{itemize}
    \item In development for ~30 years, mainly in France (INRIA, \'{E}cole
          Polytechnique, University of Paris-Sud, Paris Diderot University,
          CNRS, \'{E}cole Normale Supérieure de Lyon)
    \item Dependently-typed programming language
    \item Logic based on the Calculus of Constructions
    \item Curry-Howard Isomorphism: Type systems and logics are equivalent
  \end{itemize}
\end{frame}
\begin{frame}
  \frametitle{Our Formalization}
  \begin{itemize}
    \item Machine-checked proofs of Pottier \& Conchon's scheme
    \item ~700 lines of Coq code
    \item Relies on external Autosubst library (Sch\"{a}fer \& Tebbi 2014)
          to handle substitution over terms
    \item Incomplete
  \end{itemize}
\end{frame}
\section{End Matter}
\subsection{End Matter}
\begin{frame}
  \frametitle{Future Directions}
  \begin{itemize}
    \item Complete formalization
    \item Extend to more practical languages -- mutable references, exceptions,
          concurrency...
      \begin{itemize}
        \item Pottier \& Simonet (2002) reject label-based approach for full
              ML: ``It appears difficult for a labelled semantics to account
              for the effect of code that is \emph{not} executed; so the
              approach must be reconsidered...''
        \item Monadic transformations?
        \item Region/effect/closure-based type?
      \end{itemize}
    \item Principled disclosure
    \item ???
  \end{itemize}
\end{frame}
\begin{frame}[allowframebreaks]
  \frametitle{Bibliography}
  \begin{thebibliography}
    \beamertemplatearticlebibitems
    \bibitem{Pottier2000}
      Pottier, Fran\c{c}ois, and Sylvain Conchon.
      \newblock ``Information Flow Inference For Free''.
      \newblock {\em ACM Notices} 35.9 (2000): 46-57.
    \beamertemplatebookbibitems
    \bibitem{Pierce2002}
      Pierce, Benjamin C.
      \newblock {\em Types and Programming Languages}.
      \newblock Cambridge, MA: MIT, 2002.
    \beamertemplatearticlebibitems
    \bibitem{Sabelfeld2003}
      Sabelfeld, Andrei, and Andrew C. Myers.
      \newblock ``Language-Based Information Flow Security''.
      \newblock {\em IEEE Journal on Selected Areas in Communications}
                21.1 (2003): 5-19.
    \beamertemplatearticlebibitems
    \bibitem{Volpano1996}
      Volpano, Dennis, Geoffrey Smith, and Cynthia Irvine.
      \newblock ``A Sound Type System for Secure Flow Analysis''.
      \newblock {\em Journal of Computer Security} 4.3 (1996): 167-87.
    \beamertemplatearticlebibitems
    \bibitem{Heintze1998}
      Heintze, Nevin, and John G. Riecke.
      \newblock ``The SLam Calculus: Programming With Secrecy And Integrity''.
      \newblock {\em POPL '98 Proceedings of the 25th ACM SIGPLAN-SIGACT
                     Symposium on Principles of Programming Languages}
                (1998): 365-77.
    \beamertemplatearticlebibitems
    \bibitem{Pottier2002}
      Pottier, Fran\c{c}ois, and Vincent Simonet.
      \newblock ``Information Flow Inference For ML''.
      \newblock {\em ACM SIGPLAN Notices} 37.1 (2002): 319-30.
    \beamertemplatearticlebibitems
    \bibitem{Pottier2002}
      Sch\"{a}fer, Steven, and Tobias Tebbi.
      \newblock {\em Autosubst: Automation for de Bruijn syntax and
                     substitution in Coq}
      \newblock Web.
  \end{thebibliography}
\end{frame}
\begin{frame}
  \frametitle{Questions?}
\end{frame}
\end{document}
